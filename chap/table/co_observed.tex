\begin{flushleft}
\begin{deluxetable}{lcccccccccc}
\tabletypesize{\tiny} \tablecolumns{10} \tablewidth{0pc}
\tablecaption{Observed Parameters of $^{13}$CO Cores \label{tbl1}}
\tablehead{
\colhead{} &\colhead{}&\colhead{}&\colhead{} & \multicolumn{3}{c}{$^{12}$CO (1-0)} &\multicolumn{3}{c}{$^{13}$CO (1-0)} \\
\cline{5-7} \cline{8-10} \\
\colhead{Name} & \colhead{R.A.\tablenotemark{a}}   & \colhead{DEC.\tablenotemark{a}}    & \colhead{$R_{eff}$\tablenotemark{a}} &\colhead{$V_{\mathrm{LSR}}$\tablenotemark{b}} &
\colhead{$\triangle V (^{12}\mathrm{CO})$\tablenotemark{b}}  & \colhead{$T_{\mathrm{mb}}(^{12}\mathrm{CO})$\tablenotemark{b}}& \colhead{$V_{\mathrm{peak}}$\tablenotemark{b}} &
\colhead{$\triangle V (^{13}\mathrm{CO})$\tablenotemark{b}}  & \colhead{$T_{\mathrm{mb}}(^{13}\mathrm{CO})$\tablenotemark{b}} \\
\colhead{ } & \colhead{(h m s)}   & \colhead{(d m s)}   & \colhead{(arcmin)} &\colhead{(km s$^{-1}$)} & \colhead{(km s$^{-1}$)}  & \colhead{(K)}&
\colhead{(km s$^{-1}$)} & \colhead{(km s$^{-1}$)}  & \colhead{(K)}
}
\startdata
PMO-A & 18:36:40.58 & -6:38:44.20 & 1.6 & 41.29(0.02) & 2.39(0.05) & 30.0 & 41.28(0.01) & 1.77(0.03) & 11.5 \\
PMO-B & 18:36:47.78 & -6:35:56.20 & 1.9 & 40.36(0.03) & 2.29(0.06) & 19.4 & 40.3 (0.02) & 1.73(0.05) & 7.2 \\
PMO-C & 18:36:28.58 & -6:38:44.20 & 1.2 & 41.40(0.04) & 2.77(0.10) & 20.0 & 41.81(0.02) & 1.92(0.06) & 6.9 \\
PMO-D & 18:36:16.58 & -6:37:56.20 & 0.7 & 40.51(0.03) & 3.04(0.08) & 8.5  & 40.66(0.1)  & 2.65(0.21) & 1.7 \\
\enddata
\tablenotetext{a}{Parameters resulting from  the Clumpfind algorithm; Of these, $R_{eff}=\sqrt{(4A/\pi-\theta_{b}^{2})}/2$ is the deconvolved effective radius, where A is the projected area of
each core and $\theta_{b}$ is the beam width.}
\tablenotetext{b}{Parameters derived from the Gauss fitting in GILDAS.}
\end{deluxetable}
\end{flushleft}